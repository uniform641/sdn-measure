\chapter{实验分析}
\label{cha:analyze}

\section{跳数}

\subsection{传统路由的情况}

由于在传统路由中,路由条数是完全确定的,
且由于这里使用的图是无向图,因而路由是完全对称的。
这样一来我们只需要分析$u_1$到$u_3$的路径和$u_5$到$u_6$的路径,
即可确定跳数。

经过对图的分析,知路径和跳数如下:

% \usepackage{booktabs}
% \usepackage{caption}

\begin{table}[h]
  \centering
  \begin{tabular}{ccc}
    \toprule
    & 跳数 & 路径 \\
    \midrule
    $ u_1 - u_3 $ & 3 & $ u_1 \rightarrow s_1 \rightarrow s_2 \rightarrow s_3 \rightarrow u_3 $ \\
    $ u_5 - u_6 $ & 3 & $ u_1 \rightarrow s_5 \rightarrow s_4 \rightarrow s_3 \rightarrow u_6 $ \\
    \bottomrule
  \end{tabular}
  \caption{传统路由情况下的条数路径表}
\end{table}

\subsection{SDN路由的情况}

在SDN路由中,我们更改了TCP和UDP的路由。故此处要分析四种路径:

% \usepackage{booktabs}
% \usepackage{caption}

\begin{table}[h]
  \centering
  \begin{tabular}{ccc}
    \toprule
    & 跳数 & 路径 \\
    \midrule
    $ u_1 - u_3(\texttt{TCP}) $ & 3 & $ u_1 \rightarrow s_1 \rightarrow s_2 \rightarrow s_3 \rightarrow u_3 $ \\
    $ u_1 - u_3(\texttt{UDP}) $ & 4 & $ u_1 \rightarrow s_1 \rightarrow s_6 \rightarrow s_4 \rightarrow s_3 \rightarrow u_3 $ \\
    $ u_5 - u_6(\texttt{TCP}) $ & 4 & $ u_1 \rightarrow s_5 \rightarrow s_6 \rightarrow s_2 \rightarrow s_3 \rightarrow u_6 $ \\
    $ u_5 - u_6(\texttt{UDP}) $ & 3 & $ u_1 \rightarrow s_5 \rightarrow s_4 \rightarrow s_3 \rightarrow u_6 $ \\
    \bottomrule
  \end{tabular}
  \caption{传统路由情况下的条数路径表}
\end{table}

\subsection{对比}

从上面的两个表格可以很清楚地看到,使用SDN反而会使得某些情况下的条数变多。
但是从另外的角度上来说,如果不使用SDN,那么实际上只使用了一部分线路,没有充分利用线路资源。

\section{带宽}


\section{延迟}

\section{丢包}

